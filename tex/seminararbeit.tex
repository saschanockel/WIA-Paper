\documentclass[conference,compsoc,final,a4paper]{IEEEtran}
\usepackage[utf8]{inputenx}

%% Bitte legen Sie hier den Titel und den Autor der Arbeit fest
\newcommand{\autoren}[0]{Nockel, Sascha}
\newcommand{\dokumententitel}[0]{Das digitale Gedächtnis, warum erinnern allein nicht ausreicht}

\input{preambel} % Weitere Einstellungen aus einer anderen Datei lesen

\begin{document}

% Titel des Dokuments
\title{\dokumententitel}

% Namen der Autoren
\author{
  \IEEEauthorblockN{\autoren}
  \IEEEauthorblockA{
    Hochschule Mannheim\\
    Fakultät für Informatik\\
    Paul-Wittsack-Str. 10,
    68163 Mannheim
    }
}

% Titel erzeugen
\maketitle
\thispagestyle{plain}
\pagestyle{plain}

% Eigentliches Dokument beginnt hier
% ----------------------------------------------------------------------------------------------------------

% Kurze Zusammenfassung des Dokuments
\begin{abstract}

\end{abstract}

% Inhaltsverzeichnis erzeugen
{\small\tableofcontents}

% Abschnitte mit \section, Unterabschnitte mit \subsection und
% Unterunterabschnitte mit \subsubsection
% -------------------------------------------------------
\section{Einleitung}


% -------------------------------------------------------
\section{Digitale Archive}
Seit Menschen damit begonnen haben Wissen und Geschichten für folgende Generationen festzuhalten, sei es auf Steintafeln, Schriftrollen oder in Büchern, ist der Wissensvorrat der Menschheit stetig gewachsen. Dank diesem Wissensvorrat wird es uns heute ermöglicht, komplizierte Sachverhalte und bedeutende Errungenschaften großer Denker und Erfinder der Vergangenheit nachzuvollziehen und auf ihnen aufzubauen, ohne dass jeder das Sprichwörtliche Rad neu erfinden muss um beispielsweise ein Auto zu bauen.

All dieses Wissen, genauer die Medien auf denen es festgehalten ist, muss irgendwo gelagert und verwaltet werden, da es niemandem von Nutzen ist wenn man nicht finden kann wonach man sucht. So ist schon in der Antike das wohl bekannteste Archiv der Welt entstanden, die Bibliothek von Alexandria. Heutige Archive und Bibliotheken bestehen noch immer zu großen Teilen aus schriftlichen Medien, seit dem Beginn des Zeitalters der Digitalisierung werden diese jedoch immer mehr von den digitalen Datenträgern wie CDs, Festplatten oder Magnetbändern abgelöst, da sie im Vergleich zu einem Buch bei gleichem oder geringerem Platzbedarf eine vielfach größere Menge an Daten speichern können. So ist ein eBook je nach Format zumeist gerade einmal ca. 2 \ac{MB} groß während beispielsweise handelsüblich Festplatten mit Kapazitäten von hunderten \ac{GB} bis hin zu mehreren \ac{TB} erhältlich sind. Auf einer Festplatte mit nur 3 \ac{TB} Kapazität können also bei dem physikalischen Platzbedarf eines Taschenbuches schon ca. 1,5 Millionen eBooks untergebracht werden.

Nicht nur wegen der effizienteren Speicherung der Daten, sondern auch wegen der Möglichkeit des schnelleren und einfacheren Auffindens der gesuchten Inhalte setzen Bibliotheken und Archive schon lange auf digitale Datenhaltung zusätzlich zu ihrem normalen Angebot. Da es eine immer größer werdende Menge an digital erschaffenen Inhalten gibt, wie etwa Word Dokumente oder \acs{PDF}s, nimmt die Bedeutung digitaler Archive immer weiter zu.

% -----------------------------
\subsection{Anwendungsbereiche}
Digitale Archive finden eine weite Verbreitung, denn überall wo digitale Daten anfallen ist es potentiell gewünscht diese auf lange Zeit zu speichern, dies kann im kleinen Rahmen mit der privaten Foto- und Videosammlung oder im größeren wie zum Beispiel bei Firmen stattfinden, die in Deutschland auch gesetzlich dazu verpflichtet sind einen Großteil ihrer Akten auf mehrere Jahre \autocite{AO147} und im Fall von medizinischen Daten sogar Jahrzehnte zu archivieren, für den Fall dass Schadenersatzansprüche aufgrund fehlerhafter Behandlung geltend gemacht werden \autocite{BGB852}.

Die Art der gespeicherten Daten in digitalen Archiven ist je nach Anforderung ganz unterschiedlich, so werden etwa im deutschen Bundesarchiv ausschließlich Sachakten archiviert \autocite{Berger2005}, das deutsche Satellitendatenarchiv (\acs{D-SDA}), das vom deutschen Zentrum für Luft- und Raumfahrt betrieben wird, archiviert dagegen ausschließlich Daten die bei Missionen zur Erdbeobachtung anfallen \autocite{Memishi2019}. Dies sind Beispiele für spezialisierte Archive, jedoch gibt es auch Archive die ein breiteres Spektrum an Daten beinhalten. Das gemeinnützige Internet Archive Projekt in San Francisco hat beispielsweise damit begonnen Webseiten zu archivieren und mit der eigens entwickelten Wayback Machine abrufbar zu machen. Inzwischen beinhaltet das Internet Archive jedoch auch Bücher, Videos, Audioaufnahmen, Software und Bilder \autocite{AboutIA}.

% -----------------------------
\subsection{Anforderungen}
In der Regel stellen große Archive wie das deutsche Bundesarchiv oder das \acs{D-SDA} sehr individuelle Anforderungen an ihr jeweiliges Archivsystem. So kommt es nicht selten vor, dass die Archive speziell für ihre Bedürfnisse entwickelte Software verwenden, wie zum Beispiel das Bundesarchiv welches die Eigenentwicklung BASYS-S-Oracle zur Erschließung der bereits erwähnten Sachakten verwendet, für andere Daten wie Bilder oder Audioaufnahmen ist diese Software hingegen nicht geeignet \autocite{Berger2005}.

Bei der Anforderungsanalyse spielen viele Faktoren eine Rolle, so ist für eine Bibliothek die Benutzerfreundlichkeit der Archivsoftware meist von großer Wichtigkeit, während ein reines Archiv eher darauf bedacht ist Risiken zu minimieren, was mithilfe von Replikationen und verschiedenen Redundanzmechanismen bei der Speicherung der Daten ermöglicht wird. Das Internet Archive betreibt aufgrund dieser Anforderung Spiegelserver in der neuen Bibliothek von Alexandria welche immer eine Kopie der Server in San Francisco bereithalten \autocite{BibAlexIAMirror}. Eine solche Risikominimierung ist natürlich mit erheblichen Kosten verbunden und deshalb nicht unbedingt für jedes Archiv von gleicher Wichtigkeit. Nicht nur an die Hardware, sondern auch an die Software, werden wie der Fall des deutschen Bundesarchivs zeigt je nach Einsatzgebiet besondere Anforderungen gestellt, um systematisch eine Lösung für beide Bereiche zu erarbeiten gibt es verschiedene Mehtoden die sich bewährt haben. Eine davon ist der PLANETS Preservation Planning approach \autocite{Strodl2007}, hier werden in Workshops mit Experten Zielbäume erarbeitet bei denen es mehrere hierarchische Ebenen geben kann und jedem Blatt messbare Einheiten wie etwa die Kosten pro Jahr für einen Spiegelserver und eine Gewichtung zugeordnet werden. Am Ende des Prozesses kann mathematisch eine Rangliste für die einzelnen Teilbereiche des Zielbaums erstellt werden auf dessen Grundlage die Entscheidung für eine Lösung getroffen werden kann.

% -----------------------------
\subsection{Umsetzung}
Das \ac{OAIS} Referenzmodell, das ursprünglich in der ISO 14721:2003 vorgestellt und 2012 in der ISO 14721:2012 noch einmal überarbeitet wurde, bietet eine konzeptionelle Anleitung für die Umsetzung eines digitalen Archivs. Das Informationsmodell nach \ac{OAIS} sieht es vor, dass alle Daten die durch das System fließen in Pakete gepackt werden. Die drei Arten von Paketen sind \ac{SIP}, \ac{AIP} und \ac{DIP}. Die \ac{SIP}s sind hierbei der Einstieg der Daten in das Archivsystem, da in diesen Paketen Daten und Informationen enthalten sind die benötigt werden um aus ihnen \ac{AIP}s zu erstellen, welche die Pakete sind die eigentlich im Archivsystem gespeichert werden. Die \ac{DIP}s sind wiederum Versionen von \ac{AIP}s die auf bestimmte Anforderungen der Nutzer oder anderer Systeme welche die Daten verwenden abgestimmt sind. Diese drei Pakettypen bestehen grundsätzlich aus zwei Blöcken, zum einen dem Block für die Inhaltsinformationen, der die eigentlichen Daten und die zugehörigen Metadaten enthält sowie dem \ac{PDI} Block. Der \ac{PDI} Block beinhaltet eine genaue Änderungshistorie für den Block der Inhaltsinformationen sowie Informationen über den Kontext in dem die Daten stehen und welche Referenzen zu den Daten vorhanden sind. Außerdem beinhaltet er auch Informationen über die Beständigkeit der Daten wie etwa Prüfsummen um verifizieren zu können, dass die Daten nicht korrumpiert wurden. Diese beiden Blöcke umgeben die eigentlichen Paketinformationen die das Suchen und wieder auffinden des Pakets ermöglichen und es identifizierbar machen sollen.

Neben dem Aufbau der Informationspakete werden auch Verantwortlichkeiten beschrieben die ein \ac{OAIS} wahrnehmen sollte um konform zu sein, da die Norm jedoch nicht den Anspruch erhebt eine definitive Anleitung zur Implementierung eines solchen Systems zu sein, existieren gewisse Freiheitsgrade weshalb nicht alle Archivsysteme alle Punkte genau umsetzen \autocite{Ball2006}.

% -------------------------------------------------------
\section{Herausforderungen im Lebenszyklus von Archivdaten}


% -----------------------------
\subsection{Neue Hard- und Software}


% -----------------------------
\subsection{Unbeständigkeit von Dateiformaten}


% -----------------------------
\subsection{Verschlüsselung und Sicherheit}


% -------------------------------------------------------
\section{Maßnahmen zur Erhaltung der Daten}


% -----------------------------
\subsection{Emulation}


% -----------------------------
\subsection{Migration}


% -----------------------------
\subsection{Vergleich der beiden Methoden}


% -------------------------------------------------------
\section{Fazit}


% ----------------------------------------------------------------------------------------------------------
\section*{Abkürzungen}
\addcontentsline{toc}{section}{Abkürzungen}

% Die längste Abkürzung wird in die eckigen Klammern
% bei \begin{acronym} geschrieben, um einen hässlichen
% Umbruch zu verhindern
% Sie müssen die Abkürzungen selbst alphabetisch sortieren!
\begin{acronym}[IEEE]
\acro{AIP}{Archival Information Package}
\acro{SIP}{Submission Information Package}
\acro{DIP}{Dissemination Information Package}
\acro{PDI}{Preservation Description Information}
\acro{OAIS}{Open Archival Information System}
\acro{D-SDA}{Deutsches Satellitendatenarchiv}
\acro{PDF}{Portable Document Format}
\acro{MB}{Megabyte}
\acro{GB}{Gigabyte}
\acro{TB}{Terabyte}
\end{acronym}

% Literaturverzeichnis
\addcontentsline{toc}{section}{Literatur}
\nocite{*} % Auch alle nicht im Text zitierten Quellen drucken
\printbibliography

\end{document}
